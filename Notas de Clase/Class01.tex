\section{Introducción}
Empezamos con la discusión  de la doble rendija. Empezamos con la caracterización del patrón de interferencia.
En mecanica cuantica empezamos con el objetivo de hallar una función de onda $ \phi $ que es solucion a la siguiente ecuacion de onda:
\[ i \hbar \frac{\partial \psi}{\partial t}  = - \frac{\hbar^2}{2m} \frac{\partial^2 \psi}{\partial x^2}\]
Seguimos entonces una interpretación probabilista.
\subsection{Axiomas de la mecánica cuántica}
Empezamos con la suposición que son sistemas cuanticos cerrados
\begin{itemize}
	\item \textbf{Postulado 1:} Asociado a todo sistema fisico aislado, se tiene un espacio Hilbert conocido como el espacio de estados, y el sistema esta completamente determinado. Un estado es rayo en un espacio de Hilbert.
	
\end{itemize}

\begin{define}
	Un \textbf{espacio de Hilbert} de dimension finita es un espacio vectorial complejo de dimension finita, con un producto interno.
\end{define}